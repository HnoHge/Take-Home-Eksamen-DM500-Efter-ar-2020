\documentclass[a4paper,12pt]{article}
\usepackage[utf8]{inputenc}
\usepackage{graphicx}
\graphicspath{ {images/} }
\usepackage{hyperref}
\usepackage{mathtools}
\usepackage{amsmath}
\parindent 0ex



\title{\LaTeX{}}
\author{Andreas Twisttmann Askholm, \\Mikkel Lykke Bentsen, \\Hanno Hagge}
\date{\today}

%comments are preceded by a % sign
%bold text \textbf{}
%italics \textit{}
%underlined \underline{}
%fremhæv \emph{}
%billede \includegraphics{}
%matematik \(...\), $...$ eller \begin{math}...\end{math}

\begin{document}

\maketitle
useful links:
\begin{itemize}
	\item \href{https://www.overleaf.com/learn/latex/Learn_LaTeX_in_30_minutes}{overleaf \LaTeX{} tutorial}
	\item \href{https://www.latex-tutorial.com/tutorials/hyperlinks/}{\LaTeX{} Tutorial}
\end{itemize}
Reeksamen februar 2015
\\ Opgave 1.
I det følgende lader vi $U = \lbrace1, 2, 3, . . . , 15\rbrace$ være universet (universal set).\\
Betragt de to mængder\\
$$A = {2n \mid n \in S}$$
$$B = {3n + 2 \mid n \in S}$$
hvor S = $\lbrace1, 2, 3, 4\rbrace.$ \\
Angiv samtlige elementer i hver af følgende mængder.
\\\\
a) A  \hspace{5mm} Mængden A er alle værdier i  S ganget med 2 (2n).$$ A = \lbrace 2, 4, 6, 8\rbrace$$
b) B \hspace{5mm} Mængden B er alle værdier i S ganget med 3, og derefter adderet med 2 (3n + 2). $$ B = \lbrace 5, 8, 11, 14 \rbrace$$
c) $A \cap B$ \hspace{5mm} Fællesmængden af A og B er den mængde bestående af de elementer de har tilfælles. $$ A \cap B = \lbrace8\rbrace$$
d) $A \cup B$ \hspace{5mm} Foreningsmængden af A og B er mængden bestående af alle elementer fra A og B. Det samme element kan ikke optræde flere gange. $$ A \cup B = \lbrace 2, 4, 5, 6, 8, 11, 14\rbrace$$
e) $A - B$ \hspace{5mm} Mængden A - B er den mængden A uden de elementer A har tilfælles med B. $$ A - B = \lbrace 2, 4, 6 \rbrace $$
f) $ \bar{A}$ \hspace{5mm} Komplimentet af A er bestående af alle de elementer i universet som \emph{ikke} er i A. $$ \bar{A} = \lbrace 1, 3, 5, 7, 9, 10, 11, 12, 13, 14, 15 \rbrace$$

\end{document}